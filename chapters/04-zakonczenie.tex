
\chapter{Zakończenie}

Zakończenie pracy zwane również Uwagami końcowymi lub Podsumowaniem powinno zawierać ustosunkowanie
się autora do zadań wskazanych we wstępie do pracy, a w szczególności do celu i zakresu pracy oraz
porównanie ich z faktycznymi wynikami pracy. Podejście takie umożliwia jasne określenie stopnia
realizacji założonych celów oraz zwrócenie uwagi na wyniki osiągnięte przez autora w ramach jego
samodzielnej pracy.

Integralną częścią pracy są również dodatki, aneksy i załączniki zawierające stworzone w ramach pracy programy, aplikacje i projekty.
