\chapter{Wstęp}

Szybka ewolucja wirtualnych światów i rosnąca integracja technologii immersyjnych w codziennym życiu wzbudzają zainteresowanie coraz większej liczby użytkowników, którzy coraz intensywniej interesują się cyfrowym wszechświatem łączącym rzeczywistość rozszerzoną i wirtualną ze światem fizycznym. Złożony charakter tych wirtualnych środowisk stwarza jednak poważne wyzwania w zakresie zarządzania zasobami, w tym alokacji, optymalizacji i skalowalności zasobów obliczeniowych. \\

Rozwój technologii wirtualnej rzeczywistości oraz wzrost popularności środowisk Metaverse stworzyły nowe wyzwania związane z zarządzaniem zasobami w środowiskach rozproszonych. Metaverse, jako złożone środowisko cyfrowe, w którym użytkownicy mogą wchodzić w interakcje, tworzyć i wymieniać zasoby, wymaga zaawansowanych mechanizmów zarządzania, aby zapewnić bezproblemowe i responsywne doświadczenia. W szczególności, dynamiczny charakter Metaverse sprawia, że zapotrzebowanie na zasoby może się gwałtownie zmieniać, co wymaga elastycznych i skalowalnych rozwiązań.\\

W ramach pracy postanowiono zaadresować problem przydziału zasobów w środowisku Metaverse. Dlatego, podstawowym celem niniejszej pracy magisterskiej jest opracowanie i implementacja systemu doboru zasobów, który efektywnie alokuje zasoby w odpowiedzi na zapotrzebowanie użytkowników w środowisku Metaverse. W kontekście dynamicznie zmieniających się wymagań użytkowników i dostępnych zasobów, system ten ma na celu zapewnienie optymalnego przydziału zasobów, minimalizując opóźnienia i maksymalizując wydajność oraz zapewnienie skalowalności systemu.\\

Rozdział pierwszy zawiera cel pracy, kontekst, motywację oraz zawartość kolejnych rozdziałów. Rozdział drugi zawiera teoretyczne zagadnienia dotyczące Metaverse. W rozdziale trzecim opisana została literatura wykorzystywana podczas pisania pracy. Rozdział czwarty zawiera ogólną koncepcję tworzonego systemu wraz ze schematami projektowanego systemu i schematem wymiany wiadomości między komponentami systemu. Rozdział piąty zawiera opis narzędzi wykorzystywanych do implementacji systemu.Rozdział szósty zawiera szczegółowy opis elementów zaimplementowanego systemu oraz opis  w jaki sposób system jest uruchamiany. Rozdział siódmy zawiera opis przeprowadzonych testów wydajnościowych systemu. Rozdział ósmy zawiera wnioski wynikające z pracy oraz koncepcje rozwoju zaprojektowanego systemu.