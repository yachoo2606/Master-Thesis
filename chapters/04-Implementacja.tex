\chapter{Implementacja}
\section{Wybrane narzędzia i technologie}
\subsection{Java}

\begin{figure}[!htbp]
    \centering
    \includegraphics[width=0.2\textwidth]{images/javaLogo.png}
    \caption{Oficjalne logo języka Java}
    \label{fig:enter-label}
\end{figure}


Java jest to szeroko stosowany obiektowy język programowania i platforma oprogramowania, która działa na miliardach urządzeń. Zasady oraz składnia języka zostały oparte na językach \acronym{C} i \acronym{C++}. Java powstała aby udoskonalić i naprawić wiele błędnych konceptów wprowadzonych przez te języki. Jedną z głównych zalet tworzenia oprogramowania w Javie jest jej przenoszalność. Po napisaniu kodu programu na jednym urządzeniu można go łatwo przenieść na inne urządzenie o innej architekturze. Język ten został wynaleziony w 1995 roku przez Jamesa Goslinga z Sun Microsystems (później przejętego przez Oracle), główną ideą jego wynalezienia była możliwość, \definicja{"napisania raz, uruchomienia w dowolnym miejscu"}  (\english{write once, run anywhere}). Nowe i ulepszone narzędzia do tworzenia oprogramowania pojawiają się na rynku w nizwykłym tempie, wypierając dotychczasowe produkty, które kiedyś uważano za niezbędne. W świetle tej ciągłej rotacji, długowieczność Javy jest imponująca, prawie trzy dekady po jej stworzeniu, Java jest nadal najpopularniejszym językiem do tworzenia oprogramowania użytkowego\cite{javaIBM}\cite{javaDEV}.

Wszystkie języki programowania służą do komunikacji z maszynami. Sprzęt maszynowy reaguje tylko na komunikację elektroniczną. Języki programowania wyskogieko poziomu, takie jak Java, działają jako pomost między językiem ludzkim a językiem sprzętu. Aby korzystać z języka Java programista musi mieć świadomość o dwóch poziomach abstrakcji pisanych programów. 
\begin{itemize}
    \item Język Java i interfejsy API \footnote{Interfejs programowania aplikacji (\english{application programming interface, \acronym{API}})}
    \item Wirtualna maszyna Java \footnote{\english{\termdef{Java Virtual Machine} \acronym{JVM}}}
\end{itemize}

Java definiuje składnię i semantykę języka. Obejmuje to podstawowe słownictwo i reguły używane do pisania algorytmów. 
Interfejsy API są ważnymi komponentami oprogramowania dołączonymi do platformy Java. Są to wstępnie napisane programy, które można podłączyć i odtworzyć istniejące funkcjonalności we własnym kodzie. Na przykład można użyć interfejsów API Java, aby uzyskać datę i godzinę, wykonać operacje matematyczne lub manipulować tekstem. Każdy kod aplikacji Java napisany przez programistę zazwyczaj łączy nowy i wcześniej istniejący kod z interfejsów API Java i bibliotek\cite{javaAmazon}\cite{javaDEV}.

Wirtualna maszyna Javy działa jako dodatkowa warstwa abstrakcji między platformą Java a sprzętem maszyny. Kod źródłowy Java może działać tylko na tych maszynach na których zainstalowano JVM. Kiedy po raz pierwszy opracowano języki programowania, dzieliły się one na dwie szerokie kategorie, w zależności od tego, w jaki sposób komunikowały się ze sprzętem: 

\begin{itemize}
    \item Kompilowany - program jest napisany w składni języka, a następnie kompilator tłumaczy cały kod na kod maszynowy. Skompilowany kod jest następnie uruchamiany na sprzęcie.
    \item Interpretowany - Dzięki interpreterom każda instrukcja kody wysokiego poziomu jest na bieżąco interpretowana na kod maszynowy. Napisane instrukcji są natychmiast uruchamiane przez sprzęt przed przejściem do następnej instrukcji
\end{itemize}

Język Java był pierwszym językiem, który połączył obie powyższe metody, przy użyciu JVM. Każdy plik programu jest najpierw kompilowany do kodu bajtowego (\english{bytecode}). Kod bajtowy Java może być uruchamiany tylko w maszynie JVM. Następnie JVM interpretuje kod bajtowy, aby uruchomić go na podstawowej platformie sprzętowej. Jeżeli aplikacja działa na komputerze z systemem Windows, maszyna JVM zinterpretuje ją dla systemu Windows. Natomiast jeżeli aplikacja uruchomiona jest na platformie open-source, takiej jak Linux, JVM zinterpretuje ją dla systemu Linux\cite{javaAmazon}\cite{javaDEV}.

\subsection{Spring boot}

\begin{figure}
    \centering
    \includegraphics[width=0.5\textwidth]{images/springBootLogo.png}
    \caption{Logo Spring Boot framework} \footnote{Szkielet do budowy aplikacji, który dostarcza niezbędnych bibliotek i komponentów oraz definiuje strukturę i działanie danej aplikacji\cite{frameworkDef}}
    \label{fig:enter-label}
\end{figure}


\subsection{Netflix Eureka}
\subsection{Docker}
\subsubsection{Docker Compose}
\subsection{Środowisko programistyczne}
\subsection{Insomnia}




\section{Elementy tworzonego systemu}

\subsection{Producent}
\subsection{Rejestr usług}
\subsection{Węzeł protokołu}
\subsubsection{Podsystem monitoringu producentów}
\subsubsection{Podsystem propozycji zasobów}

\section{Uruchomienie środowiska}