\chapter{Podsumowanie}

\section{Wnioski}

Celem pracy było stworzenie systemu teleinformatycznego pozwalającego na zarządzanie zasobami w środowisku Metaverse. System miał składać się z systemu zarządzania zasobami, oraz systemu monitorowania usług znajdujących się w systemie. Udało się zrealizować całość zaplanowanych prac. W wyniku przeprowadzonych prac powstały trzy usługi w systemie zarządzania zasobami (rozproszony rejestr usług Eureka, węzeł protokołu oraz usługa producenta świadcząca usługi w systemie). System zarządzania zasobami, koordynuje wielu producentów i węzłów protokołu. Obejmuje on mechanizm wykrywania usług, zapewniający płynną interakcję między różnymi komponentami systemu. Węzły protokołu obsługują żądania od użytkowników, podczas gdy producenci świadczą usługi oraz zasoby udostępniane w środowisku Metaverse. Rejestr Eureka zapewnia wysoką dostępność i równoważenie obciążenia w różnych węzłach, zwiększając tym samym niezawodność i wydajność systemu. Podsystem monitoringu, integruje narzędzia takie jak Elasticsearch, Logstash oraz Kibana zapewniając kompleksowy monitoring wszystkich usług uruchomionych w systemie. Podsystem monitoringu przechwytuje, przechowuje i wizualizuje dane dotyczące usług producenta, węzła protokołu i rejestru Eureka. 

Zaimplementowany system spełnia wszystkie wcześniej określone założenia i wymagania, zapewniając niezbędne funkcjonalności do zarządzania środowiskiem Metaverse. Wybory projektowe i wdrożeniowe, takie jak mechanizm wykrywania usług i konteneryzacja systemu oraz wykorzystanie Elasticsearch do monitorowania, zapewniają wydajność, niezawodność i skalowalność systemu.

Niniejsza rozprawa demonstruje praktyczne zastosowanie systemów rozproszonych i chmurowych w rozwijającej się dziedzinie Metaverse, pokazując, w jaki sposób architektura modułowa i nowoczesne praktyki inżynierii oprogramowania mogą tworzyć elastyczne i wydajne rozwiązania do zarządzania zasobami.

\section{Koncepcja rozwoju}

Na podstawie projektu systemu oraz analizy koncepcyjnych rozwiązań i na podstawie istniejących narzędzi został zaimplementowany system zapewniający funkcjonalność zarządzania zasobami w środowisku Metaverse. Istnieją jednak dodatkowe funkcjonalności, których brak nie ogranicza działania systemu, jednak mogą one znacząco poszerzyć jego możliwości. Jedną z pierwszych funkcji dodanych do nowej wersji systemu będzie wdrożenie algorytmów uczenia maszynowego do analizy predykcyjnej. Pozwoli to systemowi prognozować przyszłe zapotrzebowanie na zasoby w oparciu o dane historyczne i wzorce użytkowania, optymalizując alokację zasobów i zmniejszając koszty operacyjne. Dodatkowo wprowadzone zostanie wykrywanie anomalii za pomocą modeli uczenia maszynowego. Umożliwi to systemowi identyfikację nietypowych wzorców lub anomalii w funkcjonowaniu systemu, które mogą wskazywać na potencjalne zagrożenia bezpieczeństwa lub awarie systemu. Następną funkcją będzie wprowadzenie systemu rekomendacji. System ten będzie sugerował użytkownikom odpowiednie zasoby lub treści w oparciu o ich preferencje i zachowanie w Metaverse, zwiększając komfort użytkowania. Kolejną funkcjonalnością będzie monitorowanie oparte na sztucznej inteligencji, które będzie analizować dane zapewniając proaktywne zalecenia dotyczące optymalizacji wydajności i niezawodności systemu.

Dzięki zaimplementowaniu tych zaawansowanych technologii i metodologii, system będzie ewoluować, aby stać się bardziej inteligentnym, bezpiecznym i zorientowanym na użytkownika, skutecznie zaspokajając dynamiczne potrzeby środowiska Metaverse.