
\chapter{Podstawy teoretyczne}
\section{Metaverse}


Metaverse to koncepcja w świecie technologicznym, która odnosi się do cyfrowego środowiska życia w którym konwencjonalne struktury społeczne ulegają zmianie. Jest to termin, który łączy w sobie koncepcje greckiego prefiksu „meta”, który oznacza „pełniejszy” lub „przekraczający”, oraz akronimu „Verse” oznaczającego „wszechświat”, który oznacza pojemnik czasoprzestrzenny. Idea metawersji została wprowadzona w powieści science fiction Neala Stephensona \definicja{Snow Crash} prawie 30 lat temu. Szybki rozwój technologii takich jak blockchain, wirtualna (\english{Virtual Reality}) i rozszerzona (\english{Augmented Reality}) rzeczywistość, gry, sztuczna inteligencja i Internet Rzeczy \acronym{IoT} (\english{Internet Of Things}) sprawiły, że metawersja stała się jednym z najbardziej popularnych terminów w świecie technologii. Rozwiązania i usługi są opracowywane dla wirtualnych światów, aby umożliwić użytkownikom dobrą zabawę, inteligentne angażowanie się w otoczenie i nawiązywanie głębszych relacji z innymi użytkownikami \cite{metaverseAsAService}. 

\begin{figure}[h!]
    \centering
    \includesvg[width=0.7\textwidth]{images/metaverse/MetaverseInfographic.svg}
    \caption{Koncepcyjny widok metaverse\cite{metaverseUseCaseslee}}
    \label{fig:enter-label}
\end{figure}

Metaverse jako wirtualny świat mapujący i wchodzący w interakcję ze światem rzeczywistym, jest uważany za idealne ucieleśnienie Internetu w przyszłości. Zintegrowany z zaawansowanymi technologiami, metawersja może być wirtualną przestrzenią wzbogaconą o rzeczywistość fizyczną. Użytkownicy są połączeni w wirtualnym wszechświecie we wciągającej interakcji i są ze sobą połączeni w celu prowadzenia działań społecznych. Odkąd koncepcja metaverse została zaproponowana w filmie naukowym \definicja{Snow Crash}, ludzie są stopniowo przyzwyczajają się do wirtualnych i internetowych aktywności zamiast w fizyczny i konwencjonalny sposób. Odkąd zaproponowano Przemysł 4.0, produkcja przemysłowa przekształca się w kierunku inteligentnej produkcji zwłaszcza w całym cyklu życia produktu, obejmującym badania i rozwój, produkcję, testy i eksperymenty, sprzedaż i transakcje oraz usługi i konserwację. Dzięki zaawansowanym technologiom informacyjno-komunikacyjnym, technologiom rozszerzonej rzeczywistości i sztucznej inteligencji, inteligentna produkcja jest wspierana przez wyższą wydajność produkcji i zwiększoną wirtualną interaktywność użytkowników. Ponadto, jako nowy paradygmat produkcji, produkcja w chmurze wygodnie zapewnia użytkownikom usługi na żądanie. Rozproszone zasoby produkcyjne są wirtualizowane i zarządzane w ujednolicony, zoptymalizowany i konfigurowalny sposób, umożliwiając wysoce wirtualną współpracę i innowacyjną produkcję\cite{industrialMetaverseForSmartManufacturing}. 


Kluczowe cechy metaverse:

\begin{itemize}
    \item Trwałość oznacza, istnienie niezależnie od fizycznej obecności użytkownika\cite{metaverseUseCaseslee}.
    \item Nieskończoność obsługiwanie niezliczonej liczby użytkowników i światów VR\cite{metaverseUseCaseslee}.
    \item Samowystarczalność oznacza, że użytkownicy mogą zarabiać w Metaverse i płacić za swoją użyteczność\cite{metaverseUseCaseslee}.
    \item Interoperacyjność pomaga użytkownikom przenosić ich wirtualne przedmioty, w tym awatary, z jednego projektu Metaverse do drugiego\cite{metaverseUseCaseslee}.
    \item Czas rzeczywisty pozwala użytkownikom cieszyć się doświadczeniami na żywo\cite{metaverseUseCaseslee}.
\end{itemize}


Metaverse ma być wciągającym wirtualnym światem, który płynnie łączy sferę fizyczną i cyfrową, umożliwiając użytkownikom interakcję, współpracę i angażowanie się w szereg działań we wspólnym wirtualnym środowisku. U podstaw tej rewolucyjnej koncepcji leży solidna infrastruktura, która służy jako szkielet, ułatwiając płynną łączność i interoperacyjność. Infrastruktura metaverse to ewoluująca, złożona sieć wzajemnie połączonych technologii, protokołów i ram, które współpracują ze sobą w celu stworzenia jednolitego i spójnego wirtualnego wszechświata\cite{metaverseInfrastructureIEEE}.

Infrastruktura ta obejmuje szeroką gamę komponentów, w tym szybkie sieci, potężne zasoby obliczeniowe, zaawansowane urządzenia sprzętowe i najnowocześniejsze platformy oprogramowania. Wykorzystuje ona najnowsze osiągnięcia w takich dziedzinach jak rzeczywistość wirtualna, rzeczywistość rozszerzona, blockchain i zdecentralizowane przetwarzanie danych, aby zapewnić wciągające i bezpieczne doświadczenie metawersji użytkownikom na całym świecie\cite{metaverseInfrastructureIEEE}.

\subsubsection{Architektura sieci w metawersji}

Architektura sieci w metaverse została zaprojektowana tak, aby wspierać płynne interakcje i komunikację w czasie rzeczywistym, umożliwiając użytkownikom angażowanie się w różne działania bez doświadczania opóźnień lub rozłączeń. Wykorzystuje ona zaawansowane protokoły i technologie sieciowe, które priorytetowo traktują niskie opóźnienia, wysoką przepustowość i wydajny transfer danych\cite{metaverseInfrastructureIEEE}.

Zdecentralizowane sieci odgrywają kluczową rolę w zwiększaniu łączności metaverse. Wykorzystując technologię blockchain i sieci peer-to-peer, infrastruktura metaverse ma na celu zmniejszenie zapotrzebowania na centralizacje zarządców lub pośredników. Takie podejście może zapewnić odporność, przejrzystość i demokratyczne zarządzanie, umożliwiając użytkownikom udział w rozwoju metawersji i procesach decyzyjnych\cite{metaverseInfrastructureIEEE}.

Przesyłanie danych i zarządzanie nimi w ramach infrastruktury metaverse jest ułatwione dzięki połączeniu tradycyjnych technologii sieciowych i nowych systemów rozproszonych. Szybkie sieci światłowodowe i łączność bezprzewodowa 5G/6G zapewniają przepustowość niezbędną do przesyłania bogatych treści multimedialnych i strumieni danych w czasie rzeczywistym. Jednocześnie zdecentralizowane rozwiązania pamięci masowej, takie jak rozproszone systemy plików i InterPlanetary File System \akronim{IPFS}, zapewniają bezpieczne i redundantne przechowywanie danych, umożliwiając efektywny dostęp do zasobów cyfrowych i ich wyszukiwania\cite{metaverseInfrastructureIEEE}.

Technologie przetwarzania na krawędzi (\english{Edge Computing}) mogą znacząco przyczynić się do zwiększenia wydajności infrastruktury metaverse. Przybliżając zasoby obliczeniowe do urządzeń brzegowych (takich jak zestawy słuchawkowe VR i okulary AR), przetwarzanie brzegowe zmniejsza opóźnienia i poprawia szybkość reakcji, zwiększając ogólne wrażenia użytkownika. Podejście to odciąża również scentralizowane serwery od zadań przetwarzania, rozkładając obciążenie obliczeniowe na całą sieć i zapewniając skalowalność w miarę wzrostu rozmiaru i złożoności metaverse\cite{metaverseInfrastructureIEEE}.

Technologie blockchain mogą odgrywać kluczową rolę w zwiększaniu bezpieczeństwa sieci w metawersji. Wykorzystując zdecentralizowane mechanizmy konsensusu i protokoły kryptograficzne, blockchain zapewnia integralność i niezmienność danych, chroniąc przed manipulacją i nieautoryzowanym dostępem. Dodatkowo, inteligentne kontrakty ułatwiają bezpieczne i przejrzyste interakcje, automatyzując złożone procesy i umożliwiając transakcje bez zaufania w ramach architektury metaverse i modelowania 3D\cite{metaverseInfrastructureIEEE}.

\subsubsection{Wymagania sprzętowe dla metawersji}

Aby uzyskać dostęp i w pełni doświadczyć metawersji, użytkownicy będą potrzebować specjalistycznego sprzętu, który może obsługiwać wciągające środowiska wirtualne i płynne interakcje. Podstawą tych wymagań sprzętowych są urządzenia komputerowe, takie jak komputery stacjonarne, konsole do gier lub wyspecjalizowane stacje robocze do rozwoju metaverse. Urządzenia te muszą posiadać wystarczającą moc obliczeniową, możliwości graficzne i zasoby pamięci, aby renderować szczegółowe środowiska 3D i obsługiwać złożone symulacje technologii metaverse\cite{metaverseInfrastructureIEEE}.

Infrastruktura metaverse w dużym stopniu wykorzystuje urządzenia rzeczywistości rozszerzonej i wirtualnej, aby zapewnić użytkownikom wciągające wrażenia. Zestawy VR, takie jak Oculus Rift, HTC Vive i PlayStation VR, przenoszą użytkowników do w pełni zrealizowanych wirtualnych światów, umożliwiając im interakcję z cyfrowymi obiektami i awatarami tak, jakby były prawdziwe. Urządzenia AR, takie jak Microsoft HoloLens i Magic Leap One, płynnie łączą elementy cyfrowe ze światem fizycznym, umożliwiając użytkownikom rozszerzenie otoczenia o wirtualne nakładki i interaktywne interfejsy\cite{metaverseInfrastructureIEEE}.

Zaawansowane jednostki przetwarzające, w tym wysokowydajne procesory graficzne \akronim{GPU} (\english{Graphics Processing Unit}) i wyspecjalizowane akceleratory, odgrywają kluczową rolę w zwiększaniu doznań płynących z metaverse. Komponenty te są odpowiedzialne za renderowanie złożonych środowisk 3D, symulację fizyki i przetwarzanie ogromnych ilości danych w czasie rzeczywistym. Dodatkowo, integracja sztucznej inteligencji i technologii uczenia maszynowego w infrastrukturze metaverse wymaga potężnych zasobów obliczeniowych, aby umożliwić inteligentne interakcje, przetwarzanie języka naturalnego i realistyczne awatary\cite{metaverseInfrastructureIEEE}.

Wraz z ciągłym rozwojem technologii, infrastruktura metaverse dostosowuje się do nowych technologii, które mogą jeszcze bardziej poprawić wrażenia użytkownika. Przykładowo, rozwój interfejsów mózg-komputer \akronim{BCI} (\english{Brain-Computer Interface}) i haptycznych urządzeń sprzężenia zwrotnego może zrewolucjonizować sposób interakcji użytkowników ze środowiskami wirtualnymi, umożliwiając bardziej intuicyjne i wciągające doświadczenia. Co więcej, postępy w dziedzinie obliczeń kwantowych i przetwarzania fotonicznego mogą potencjalnie zrewolucjonizować metawersję, zapewniając bezprecedensową moc obliczeniową a co za tym idzie możliwości przetwarzania danych\cite{metaverseInfrastructureIEEE}.

\subsubsection{Transakcje w metawersji}

Wirtualne waluty i technologia blockchain znajdują się w czołówce, jeśli chodzi o ułatwianie transakcji w metaverse. Te cyfrowe aktywa, często określane jako kryptowaluty lub tokeny, mogą służyć jako podstawowy środek wymiany towarów, usług i wirtualnych aktywów w wirtualnym środowisku metaverse. Wykorzystując zdecentralizowany i bezpieczny charakter technologii blockchain, te wirtualne waluty umożliwiają płynne, przejrzyste i pozbawione zaufania transakcje, eliminując potrzebę pośredników i zmniejszając koszty transakcji\cite{metaverseInfrastructureIEEE}.

Inteligentne kontrakty, które są samowykonalnymi umowami zakodowanymi w sieciach blockchain, mogą odgrywać kluczową rolę w automatyzacji i zabezpieczaniu transakcji w metaverse. Kontrakty te definiują zasady i warunki różnych umów, takich jak transfery aktywów, umowy o świadczenie usług i rozliczenia finansowe. Po wdrożeniu, inteligentne kontrakty wykonują się automatycznie, gdy spełnione są wcześniej określone warunki, zapewniając przejrzystość, wydajność i niezmienne prowadzenie dokumentacji dla wszystkich transakcji w ramach immersyjnego doświadczenia metaverse\cite{metaverseInfrastructureIEEE}.

Aktywami cyfrowymi i własnością intelektualną można zarządzać poprzez połączenie technologii blockchain i zdecentralizowanych rozwiązań w zakresie przechowywania. Tokeny niewymienialne \akronim{NFT} (\english{Non-Fungible Token}) mogą być wykorzystywane do reprezentowania unikalnych zasobów cyfrowych, takich jak wirtualne nieruchomości, dzieła sztuki, przedmioty kolekcjonerskie i przedmioty w grze. Tokeny te są przechowywane w sieciach blockchain, zapewniając łatwą weryfikację własności i pochodzenia. Ponadto zdecentralizowane systemy przechowywania plików, takie jak IPFS, umożliwiają bezpieczne i redundantne przechowywanie treści cyfrowych, zapewniając długowieczność i dostępność wirtualnych zasobów\cite{metaverseInfrastructureIEEE}.

Infrastruktura metaverse może ułatwiać wirtualne rynki i interakcje gospodarcze poprzez integrację zdecentralizowanych aplikacji \akronim{dApps} (\english{Decentralized application}) i zdecentralizowanych protokołów finansowych \akronim{DeFi} (\english{decentralized finance}). Platformy te umożliwiają użytkownikom kupowanie, sprzedawanie, handlowanie i inwestowanie w szeroką gamę wirtualnych aktywów, towarów i usług. Inteligentne kontrakty automatyzują i zarządzają tymi transakcjami, zapewniając uczciwość, przejrzystość i przestrzeganie wcześniej określonych zasad. Co więcej, zdecentralizowane autonomiczne organizacje \akronim{DAO} (\english{Decentralized autonomous organization}) pozwalają na zbiorowe podejmowanie decyzji i zarządzanie w ramach tych wirtualnych rynków, wspierając poczucie wspólnoty i współwłasności\cite{metaverseInfrastructureIEEE}.

\subsubsection{Podsumowanie}

Metawersja stanowi rewolucyjny skok w sposobie, w jaki ludzie postrzegają środowiska cyfrowe i wchodzą z nimi w interakcję, zacierając granice między sferą fizyczną i wirtualną. U podstaw tej koncepcji leży wiele koncepcji solidnej i zaawansowanej infrastruktury, która posłuży jako podstawa płynnej łączności, wciągających doświadczeń i nieograniczonych możliwości. 

W miarę jak metawersja będzie ewoluować i zyskiwać popularność, jej infrastruktura będzie odgrywać kluczową rolę w kształtowaniu przyszłości cyfrowych interakcji, handlu i kontaktów społecznych. Wykorzystując najnowocześniejsze technologie, takie jak blockchain, rzeczywistość wirtualna, rzeczywistość rozszerzona i zdecentralizowane przetwarzanie, infrastruktura metaverse umożliwia bezpieczne, przejrzyste i interoperacyjne przestrzenie wirtualne.

Pomyślne wdrożenie i rozwój metawersji będzie jednak również wymagać sprostania krytycznym wyzwaniom związanym z zarządzaniem, regulacjami, bezpieczeństwem i prywatnością. Współpraca między zainteresowanymi stronami, w tym deweloperami, decydentami i szerszą społecznością, jest niezbędna do ustanowienia skutecznych ram, które sprzyjają innowacjom, jednocześnie chroniąc użytkowników i utrzymując standardy etyczne.


