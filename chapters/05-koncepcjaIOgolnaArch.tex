\chapter{Koncepcja i architektura systemu}

Niniejszy rozdział opisuje architekturę i koncepcję systemu zaprojektowanego do obsługi środowiska Metaverse poprzez integrację różnych technologii. Zaproponowany system ułatwia dynamiczne wykrywanie usług, przetwarzanie danych i monitorowanie w czasie rzeczywistym zachowania systemu. Celem jest stworzenie skalowalnego i interaktywnego wirtualnego świata, w którym producenci i użytkownicy mogą wchodzić w interakcje, zapewniając optymalną wydajność i niezawodność dzięki skutecznym praktykom monitorowania systemu.

System przedstawiony na rys.\ref{designedSystem} został zaprojektowany w celu wspierania dynamicznego i interaktywnego środowiska Metaverse poprzez integrację dwóch podstawowych podsystemów: skoncentrowanego na producentach i użytkownikach, oraz systemu monitorowania usług systemu i utrzymania.

% Temporarily change the margins for this page

\begin{figure}
    \centering
    \includesvg[width=\textwidth]{schemas/master—koncepcja-Dev.drawio.svg}
    \caption{Schemat tworzonego systemu}
    \label{designedSystem}
\end{figure}
% Restore the original margins



\section{Podsystem użytkowy}

Podsystem użytkowy ułatwia interakcję między producentami i użytkownikami. Producenci to podmioty, które tworzą i dostarczają wirtualne środowiska, zasoby cyfrowe lub usługi w ramach środowiska Metaverse. Z kolei użytkownicy wykorzystują stworzone zasoby, aby uczestniczyć w różnych działaniach i doświadczeniach. Każdy użytkownik uruchamia wyspecjalizowaną aplikację z unikalnym identyfikatorem, co pozwala mu na interakcję z systemem, żądanie zasobów i współpracę z producentami. Rejestr usług zapewnia, że producenci i użytkownicy mogą dynamicznie odkrywać i wchodzić ze sobą w interakcje, wspierając wydajny i skalowalny ekosystem.

\subsection{Rejestr usług}

Pierwszym centralnym elementem podsystemu użytkowego jest rejestr usług, który zarządza wykrywaniem i przechowywaniem informacji o uruchomionych usługach w sieci. Komponent ten działa jako centralny katalog, w którym każda usługa rejestruje się po uruchomieniu, podając szczegóły, takie jak jej lokalizacja i status. Inne usługi i aplikacje w sieci mogą następnie odpytywać ten rejestr, aby znaleźć i uzyskać informacje o zarejestrowanych usługach. Mechanizm ten zapewnia, że wszystkie usługi mogą dynamicznie lokalizować się nawzajem i efektywnie współdziałać, wspierając płynną komunikację i alokację zasobów w systemie.

\subsection{Producenci}

W zaprojektowanym systemie producenci są podmiotami, które tworzą i oferują różne wirtualne produkty i usługi. Producenci muszą najpierw zarejestrować się w centralnym rejestrze usług. Po zarejestrowaniu producenci stają się wykrywalni przez inne komponenty i użytkowników w systemie. Producenci są odpowiedzialni za dostarczanie swoich produktów i usług, oraz na żądanie użytkownika zapewniają, że ich oferty są dostępne dla użytkowników. Proces rejestracji i wykrywania ułatwia wydajną interakcję i wykorzystanie zasobów w środowisku Metaverse.

\subsection{Użytkownicy}

Aby uczestniczyć w przetwarzaniu i zamawiać zasoby od producentów, użytkownicy muszą uruchomić usługę węzła protokołu z unikalnym identyfikatorem. Ten unikalny identyfikator jest niezbędny do identyfikacji i zarządzania interakcjami każdego użytkownika w systemie.

\subsubsection{Funkcje i działania użytkownika}

Zaprojektowany użytkownik zapewnia następujące funkcje:

\begin{itemize}
    \item Aplikacja węzła protokołu: Użytkownicy są zobowiązani do uruchomienia usługi o nazwie protocol--worker. Aplikacja ta służy jako interfejs, za pośrednictwem którego użytkownicy wchodzą w interakcję z systemem Metaverse, umożliwiając im żądanie zasobów dostarczanych przez producentów.
    \item W momencie uruchomienia usługi zostaje jej nadany unikalny w skali systemu identyfikator. Ten identyfikator ma kluczowe znaczenie dla śledzenia działań użytkownika, zarządzania jego żądaniami i zapewnienia, że zasoby są odpowiednio przydzielane.
    \item Rejestracja usługi w serwerze usług: Aplikacja protocol--worker rejestruje się w serwerze usług przy użyciu unikalnego identyfikatora. Ten proces rejestracji sprawia, że użytkownik jest wykrywalny w systemie Metaverse.
    \item Zamawianie zasobów: Po rejestracji użytkownicy mogą zamawiać zasoby od producentów w systemie. Aplikacja protocol--worker zapewnia te interakcje, zbierając dane monitorujące producentów w celu późniejszego określenia, który z nich jest najbardziej odpowiedni dla użytkownika zamawiającego zasoby.
\end{itemize}

Na rys.\ref{wymianaWiadomosciWSystemie} przedstawiono sekwencję wymiany komunikatów w ramach systemu rezerwacji zasobów. Wykres ten szczegółowo opisuje interakcje pomiędzy elementami systemu:

\begin{itemize}
    \item Proces rozpoczyna się od zainicjowania przez użytkownika żądania rezerwacji zasobów. Żądanie to jest wysyłane do usługi  protocol--worker.
    \item Po otrzymaniu żądania użytkownika, usługa protocol--worker wysyła zapytanie do rejestru usług w celu uzyskania listy wszystkich zarejestrowanych producentów. Ten krok zapewnia, że protocol--worker ma najbardziej aktualne informacje o dostępnych producentach w systemie.
    \item Rejestr usług odpowiada na żądanie usługi protocol--worker z listą zarejestrowanych producentów. Lista ta zawiera informacje o producentach, którzy mogą potencjalnie spełnić żądanie użytkownika.
    \item Usługa protocol--worker przetwarza każdego producenta z listy, aby sprawdzić dostępność żądanych zasobów.
    \item Dla każdego producenta na liście usługa protocol--worker wysyła wiadomości w celu sprawdzenia dostępności żądanych zasobów. Wiąże się to z wysłaniem zapytania do producenta w celu ustalenia, czy może on zapewnić wymagane zasoby.
    \item Każdy producent odpowiada do protocol--worker z informacją o dostępności żądanych zasobów. Ta odpowiedź wskazuje, czy producent może spełnić żądanie użytkownika.
    \item Po zebraniu informacji o dostępności zasobów od wszystkich producentów, usługa protocol--worker przetwarza te odpowiedzi w celu określenia najlepszych opcji realizacji żądania użytkownika. Może to obejmować wybór producentów, którzy mogą najbardziej efektywnie zapewnić niezbędne zasoby.
    \item Na koniec usługa protocol--worker zwraca użytkownikowi zebrane informacje o dostępności zasobów u producentów.
\end{itemize}

\begin{figure}[h!]
    \centering
    \begin{msc}[
        title position=center,
        msc keyword=,
        draw frame=none,
        instance distance=2.5cm,
        left environment distance=0.5cm,
        right environment distance=0.5cm,
        label distance=0.3cm,
        title distance=0.5cm
        ]{Protocol Worker Resource Reservation}
            \declinst{user}{}{User}
            \declinst{protocolWorker}{}{Protocol Worker}
            \declinst{serviceDiscovery}{}{Service Discovery}
            \declinst{producer}{}{Producer}
            
            \footnotesize
            \nextlevel
            \mess{Request Resource Reservation}{user}{protocolWorker}
            \nextlevel
            \nextlevel
            \mess{Query All Registered Producers}{protocolWorker}{serviceDiscovery}
            \nextlevel
            \nextlevel
            \nextlevel
            \mess{List of Producers}{serviceDiscovery}{protocolWorker}
            \nextlevel
            \nextlevel
            \nextlevel
            \inlinestart{loop1}{loop for each Producer}{protocolWorker}{producer}
            \nextlevel
            \nextlevel
            \nextlevel
            \nextlevel
            \mess{Check Resource Availability}{protocolWorker}{producer}
            \nextlevel
            \nextlevel
            \nextlevel
            \mess{Resource Availability}{producer}{protocolWorker}
            \nextlevel
            \nextlevel
            \inlineend{loop1}
            \nextlevel
            \nextlevel
            \mess{\parbox{3cm}{Processing producers for products}}{protocolWorker}{protocolWorker}
            \nextlevel
            \nextlevel
            \nextlevel
            \nextlevel
            \nextlevel
            \mess{Return Resource Availability}{protocolWorker}{user}
        \end{msc}
    \caption{ Schemat wymiany wiadomości podczas przetwarzania żądania użytkownika}
    \label{wymianaWiadomosciWSystemie}
\end{figure}

\section{Podsystem monitoringu}

Drugi podsystem jest przeznaczony do monitorowania i utrzymywania stabilności i wydajności infrastruktury Metaverse. Podsystem ten stale gromadzi i przetwarza dzienniki zdarzeń ze wszystkich komponentów systemu, zapewniając wgląd w czasie rzeczywistym w działania systemu. Dzięki kompleksowym narzędziom wizualizacyjnym administratorzy mogą monitorować stan systemu, identyfikować potencjalne problemy i podejmować decyzje oparte na danych w celu zapewnienia optymalnej funkcjonalności. Takie podejście do monitorowania i konserwacji pomaga utrzymać niezawodne i wydajne środowisko Metaverse, wspierając płynną interakcję między producentami i użytkownikami.